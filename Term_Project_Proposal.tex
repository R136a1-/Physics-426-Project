%% LyX 2.1.2 created this file.  For more info, see http://www.lyx.org/.
%% Do not edit unless you really know what you are doing.
\documentclass{article}
\usepackage[T1]{fontenc}
\usepackage[latin9]{inputenc}
\usepackage{float}

\makeatletter
%%%%%%%%%%%%%%%%%%%%%%%%%%%%%% User specified LaTeX commands.
\usepackage{hyperref}

\makeatother

\begin{document}

\title{Physics 426: Term Project Proposal}


\author{Luke Siemens \& Stephanie Monty}

\maketitle

\section{Introduction}

Current models of Jupiter's atmosphere suggest the possibility of
convection occurring deep within the planet's atmosphere. As early
as 1977 Prinn \& Barshay suggest in their paper that the presence
of convection within the atmosphere at 1100K would explain the appearance
of carbon monoxide within the upper atmosphere, as observed in 1977
(Beer \& Taylor, 1977). More recently with observations from the Galileo
mission to Jupiter, convection was inferred from the direct observation
of long lived storms on the planet, appearing as ``zonal jets''
or ``long-lived ovals'' (Ingersoll et al. 630). The appearance and
longevity of the storms suggests an energy source within the planet
and the transfer of energy from said source to the atmosphere via
a form of convection (Ingersoll et al, 2000). With convection being
a probable process occurring within the atmosphere, an attempt to
study and become more familiar with the mechanisms associated with
convection in an Earth-based laboratory could yield important results.
An experiment will be performed in order to observe and characterize
convection cells on laboratory scales, after which the effects of
rotation and the possible induction of turbulance may also be explored.
The results of this ``table top'' convection experiment will then
be discussed in an attempt to relate any findings to the larger scale
of the Jovian atmosphere.


\section{Experimental Set-Up}


\section{Goals}

This experiment will aim to observe, study and characterize convection
within a laboratory setting. As mentioned in the above section, a
convective cell will be generated within a cylindrical vessel and
observed through the aid of high speed cameras, rheoscopic fliuds
various dyes. In order to determine the required temperature difference
between the bottom of the vessel and the top to induce convection,
a mathematical study will be made investigating the Rayleigh-Benard
Instability. This will be done using the definition of the dimensionless
Rayleigh Number, noting that the density profile of the vessel will
be treated as a free variable. Working from the definition of the
Rayleigh number.

\[
R_{a}=\frac{g\beta}{\nu\alpha}(T_{b}-T_{u})L^{3}
\]


\[
R_{a}=\frac{g\beta}{\mu k}c_{p}\rho^{2}(T_{b}-T_{u})L^{3}
\]


Noting above that all the variables excluding $T_{b}-T_{u}$ and $\rho$
are fixed constants related to the thermal and viscous characteristics
of the fluid to be used, while $L$ is the height of the vessel and
$g$ is the acceleration due to gravity. The critical Rayleigh number,
$R_{c}$, represents the value of the Rayleigh number, representing
the ratio of gravitational to viscous forces, at which convection
occurs in the system. For the solved case, in which the bottom boundary
is rigid, while the top is free, the case of this experiment, the
critical Rayleigh number is known. Thus, the temperature difference
required to induce convection may then become only dependent on the
density profile of the vessel. If the density if discrete, the temperate
difference may be deduced for each layer of different density fluids,
if the density profile is continuous the temperature difference required
may be expressed in equation (2) where density becomes a function
of height in the vessel. $\rho=\rho(L)$ (Bahrami, 2016).

\begin{equation}
(T_{b}-T_{u})=\left(\frac{\mu k}{g\beta}\right)\frac{R_{c}}{\rho^{2}L^{3}}
\end{equation}


\begin{equation}
(T_{b}-T_{u})=\left(\frac{\mu k}{g\beta}\right)\frac{R_{c}}{\rho^{2}(L)L^{3}}
\end{equation}


The density profile of the vessel will be approximated as being both
linear and discrete as both continuous and discrete density profiles
will be investigated. In reality the more easy of the two to generate
and model will be used in the final experiment to determine the required
temperature difference as given in equations (1) or (2).

A computational simulation will be created in order to investigate
the feasibility and compatibility of both modeling convection in three
dimensions and modeling convection in three dimensions via investigating
convection in two dimensions. This will be done using COMSOL, Python
and possible Fortran for any numerical analysis that may be necessary.
The goal associated with the computational aspect of the project will
be to successfully model convection in two dimensions with the possibility
of extending to three. 

Experimentally, the goal of the project will be to generate and observe
convection cells. Once convection cells have been observed in a stationary
reference frame, rotation will be introduced in an attempt to observe
and record any perturbations to convection that might occur due to
rotation. Any resultant turbulence that could could occur upon the
introduction of rotation into a convective system will also be investigated.
The effects of rotation will only be investigated upon the successful
creation of convection within a stationary reference frame. 

Utilizing the results from the mathematical, computational and experimental
components of the project an overall result as to the possibility,
or inference, of convection within Jupiter's atmosphere will be made.
This will be done in part theoretically, through noting the temperature
difference required in order to generate convection for fluids of
a specific density, or density profile, and comparing this against
the known density profile of the planets atmosphere. Experimentally,
observing convection in the system could lead to inferring the existence
of convection within Jupiter's atmosphere after considering comparable
scale lengths and temperature differences between the two systems.
If the appearance of convection is further supported through the appearance
of convection computationally, via modeling, this will lead to further
confidence in the final result.

\section{References}
\begin{enumerate}
\item Bahrami, M. ``Natural Convection.'' \textit{Simon Fraser University}.
Simon Fraser University, n.d. Web. 21 February 2016.
\item Beer, R. \& Taylor, F. ``The Abundance of Carbon Monoxide in Jupiter.''
\textit{ApJ} 221 (1978): 1100-1109. Print.
\item Ingersoll, A.P. et al. ``Moist convection as an energy source for
the large-scale motions in Jupiter's atmosphere.'' \textit{Nature}
403 (2000): 630-633. Print.
\item Prinn, R.G., Barshay, S.S. ``Carbon Monoxide on Jupiter and Implications
for Atmospheric Convection.'' \textit{AAAS} 198.4321 (1977): 1031-1034.
Print\end{enumerate}

\end{document}
